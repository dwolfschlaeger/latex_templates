%%% General settings

\input{general/ptdr-definitions} 	% official CMS definitions
% units and symbols
% most of them are defined in ptdr-definitions.tex, the ones that are missing there are defined here
\newcommand{\pb}{\ensuremath{\,\text{pb}}\xspace}
\newcommand{\MV}{\ensuremath{\,\text{MV}}\xspace}
\newcommand{\MVm}{\ensuremath{\,\text{MV\hspace{-0.16em}/\hspace{-0.08em}m}}\xspace}
\newcommand{\MHz}{\ensuremath{\,\text{MHz}}\xspace}
\newcommand{\T}{\ensuremath{\,\text{T}}\xspace}
\newcommand{\mrad}{\ensuremath{\,\text{mrad}}\xspace}
\newcommand{\mt}{\ensuremath{m_{\mathrm{T}}}\xspace}
\newcommand{\metSlash}{\ensuremath{{\not\mathrel{E}}_\mathrm{T}}} % alternative version to \ETslash, w/o spacing problem

% period and comma after formulas with some extra spacing
\newcommand{\paf}{\ .}
\newcommand{\caf}{\ ,}
\newcommand{\waf}[1]{\ \text{#1}}

% references
\AtBeginDocument{ % hyperref redefines \ref at beginning of the document
	\let\oldref\ref
	%%% PubCom recommendations are given as comments
	%%% Currently, I do not follow PubCom recommendations
	\newcommand{\refChap}[1]{\hyperref[#1]{Chapter~\oldref*{#1}}}		% 'Chapter 3'
	\newcommand{\refSec} [1]{\hyperref[#1]{Section~\oldref*{#1}}}		% 'Section 3'
	\newcommand{\refApp} [1]{\hyperref[#1]{Appendix~\oldref*{#1}}}		% 'Appendix B'
	\newcommand{\refFig} [1]{\hyperref[#1]{Figure~\oldref*{#1}}}		% 'Fig. 3'; at beginning of sentence 'Figure 3'
	\newcommand{\refTab} [1]{\hyperref[#1]{Table~\oldref*{#1}}}			% 'Table 3'
	\newcommand{\refEq}  [1]{\hyperref[#1]{Equation~(\oldref*{#1})}}% 'Eq. (3)'; at beginning of sentence 'Equation (3)'
}

%% singlets and doublets (for SM table)
\newcommand{\doublet}[2]{$\begin{pmatrix} #1 \\ #2 \end{pmatrix}_{\mathrm{L}}$}
\newcommand{\lsinglet}[1]{$\begin{array}{c} #1_\mathrm{R}^\mathrm{-}\end{array}$}
\newcommand{\qsinglet}[2]{$\begin{array}{c} #1_\mathrm{R} \\ #2_\mathrm{R}\end{array}$}
\newcommand{\doubarrc}[2]{$\begin{array}{c} #1 \\ #2  \end{array}$}
\newcommand{\doubarrl}[2]{$\hspace{-2ex}\begin{array}{l} #1 \\ #2  \end{array}$}
\newcommand{\doubarrr}[2]{$\begin{array}{r} #1 \\ #2  \end{array}\hspace{-2ex}$}
\newcommand{\singarrl}[1]{$\hspace{-2ex}\begin{array}{l} #1  \end{array}$}
\newcommand{\singarrr}[1]{$\begin{array}{r} #1  \end{array}\hspace{-2ex}$}

% defined to be equal symbol :=
\newcommand*{\defeq}{\mathrel{\vcenter{\baselineskip0.5ex \lineskiplimit0pt
                     \hbox{\scriptsize.}\hbox{\scriptsize.}}}%
                     =}
										
% ########## Hyphenations ##########
% \hyphenation{ex-am-ple}
		% personal definitions

%%%%%%%%%%%%%%%  Title page %%%%%%%%%%%%%%%%%%%%%%%%
\documentclass[11pt,twoside, openright, a4paper, pdftex, tdr]{new-cms-tdr}
\usepackage[left=29mm,right=25mm,top=25mm,bottom=14mm,includeheadfoot]{geometry}%includehead

%\usepackage{geometry} % see geometry.pdf on how to lay out the page. There's lots.
%\geometry{a4paper} % or letter or a5paper or ... etc
% \geometry{landscape} % rotated page geometry

\usepackage[latin1]{inputenc}

%% add line numbers
%\usepackage[mathlines]{lineno} % add option pagewise for new line number per page
%\linenumbers

\usepackage{ifthen}
\newboolean{bdraft}
\setboolean{bdraft}{true} %%set comments and lipsums on and off 

% define some colors
\usepackage[dvipsnames]{xcolor}
\definecolor{lightblue}{rgb}{0.85,0.85,0.92}
\definecolor{gray}{gray}{0.6}
\usepackage{color}
\definecolor{RWTHblue}{RGB}{0,84,159}%RWTH blau
\definecolor{RWTHlightblue}{RGB}{142,186,229}%RWTH hellblau
\definecolor{darkblue}{rgb}{0,0,0.5}

% lorem ipsum blind text
\usepackage{lipsum}
\newcommand{\lorem}{ \ifthenelse{\boolean{bdraft}} {\textcolor{lightblue}{\lipsum}} {} }
\setlipsumdefault{1}

% wider lines in tables and arrays
\renewcommand{\arraystretch}{1.1}

% manage "ToDo"s in text
\usepackage{todo} 

% define comments
\newcommand{\comment}[1]{ \ifthenelse{\boolean{bdraft}} {\textcolor{RedOrange}{\{#1\}}} {} }

% Give numbers to deeper levels, and show them in the TOC
\ifthenelse{\boolean{bdraft}} {\setcounter{tocdepth}{4}} {}
\ifthenelse{\boolean{bdraft}} {\setcounter{secnumdepth}{4}} {}


% width of pictures (if 2 pictures next to each other)
\newcommand{\pairwidth}{.481\textwidth}

% format captions
\usepackage[margin=10pt,skip=8pt, format=plain]{caption}
\KOMAoption{captions}{tableheading, bottombeside}

% configure default position of figures and tables
%\makeatletter
%\renewcommand{\fps@figure}{htbp}
%\renewcommand{\fps@table}{htbp}
%\makeatother

% make tables look nicer
\usepackage{booktabs}

% definition of particle names
\usepackage{general/pennames-pazo}
%\usepackage{hepnames}  % nice particle names, incompatible with mathpazo math fonts

% bibliography support
\usepackage[numbers,sort&compress]{natbib}

% Feynman graphs
%\usepackage{feynmp}
% Automize calls to mpost in TeXnicCenter
% see http://latex-community.org/forum/viewtopic.php?f=31&t=16193
\DeclareGraphicsRule{*}{mps}{*}{}
\makeatletter
\def\endfmffile{%
\fmfcmd{\p@rcent\space the end.^^J%
end.^^J%
endinput;}%
\if@fmfio
\immediate\closeout\@outfmf
\fi
\ifnum\pdfshellescape=\@ne
\immediate\write18{mpost \thefmffile}%
\fi}
\makeatother

% links within document
\usepackage[%
colorlinks, % verwende farbige Links
linkcolor=black, % Linkfarbe ist RWTH blau
citecolor=RWTHlightblue, % Zitatfarbe ist RWTH blau
bookmarks, % erstelle Bookmarks der Links
bookmarksopen, % Bookmarks werden beim Öffnen des Dokumentes ebenfalls geöffnet
bookmarksopenlevel=2,
urlcolor=black, % Hyperlinks sind RWTH blau 
bookmarksnumbered, % Bookmarks sind nummeriert
pdfborder={0 0 0},
plainpages=false,
pdfpagelabels,
% draft  % Draft-Version
final  % Endversion
]{hyperref}

\hypersetup{%
%	plainpages=false,
%	pdfpagemode=Normal,%Keine Navigatorspalte
%	pdfview=FitH,%Standard-View f�r Link
%	pdfstartview=FitH,%Start-Ansicht FitH,FitV,...
%	pdfpagelayout=TwoColumnRight,%OneColumn,TwoColumnLeft,TwoColumnRight,SinglePage
	colorlinks=true, % false: boxed links; true: colored links
%	bookmarksopen=true,
%	bookmarksnumbered=true,
%	bookmarksopenlevel=2,
%	pdfmenubar=true,
%	pdfwindowui=true,
%	pdffitwindow=true,
	linkcolor=black,
%	linkcolor=black,
%	linkbordercolor=false,%Rahmenfarbe um Links (1 0 0)Leerzeichen wichtig(R G B)
	citecolor=black,
%	citecolor=black,
	urlcolor=black,
	filecolor=darkblue
}

% create glossary
% http://ftp.uni-erlangen.de/ctan/macros/latex/contrib/glossaries/glossaries-user.pdf
% first try of options which are not necessarily optimal
% 'acronym' to obtain list of acronyms independent from glossary
% 'acronym' and 'nomain' to obtain only list of acronyms, without glossary
% 'xindy' uses a perl script to properly sort entries, including e.g. greek letters
%\usepackage[xindy,nomain,acronym,toc]{glossaries}
% set style of glossary
% here: use 2 columns, as we expect short entries (mainly acronyms)
%\usepackage{glossary-mcols}
%\setglossarystyle{mcolindex}
% set style of acronyms
%\setacronymstyle{long-short}
%\makeglossaries  % ensure glossary files are created

